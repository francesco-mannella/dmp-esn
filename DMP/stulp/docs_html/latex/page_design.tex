\hypertarget{page_design_sec_remarks}{}\subsection{General Remarks}\label{page_design_sec_remarks}
\begin{DoxyItemize}
\item Code legibility is more important to me than absolute execution speed (except for those parts of the code likely to be called in a time-\/critical context) or using all of the design patterns known to man (that is why I do not use P\+I\+M\+P\+L; it is not so legible for the uninitiated user. Also, I do not use the factory design pattern, but rather have clone() functions in classes ).\end{DoxyItemize}
\begin{DoxyItemize}
\item I learned to use Eigen whilst coding this project (learning-\/by-\/doing). So especially the parts I coded first might have some convoluted solutions (I didn't learn about Eigen\+::\+Ref til later...). Any suggestions for making the code more legible or efficient are welcome. The same goes for Python actually. So be gentle on me on this one; I myself will probably look back at this Python code in a few years and think\+: \char`\"{}\+How cute... I was just a Python baby when I coded that.\char`\"{}\end{DoxyItemize}
\begin{DoxyItemize}
\item For the organization of the code (directory structure), I went with this suggestion\+: \href{http://stackoverflow.com/questions/13521618/c-project-organisation-with-gtest-cmake-and-doxygen/13522826#13522826}{\tt http\+://stackoverflow.\+com/questions/13521618/c-\/project-\/organisation-\/with-\/gtest-\/cmake-\/and-\/doxygen/13522826\#13522826}\end{DoxyItemize}
\begin{DoxyItemize}
\item In function signatures, inputs come first (if they are references, they are const) and then outputs (if they are not const, they are inputs for sure). Exception\+: if input arguments have default values, they can come after outputs. Virtual functions should not have default function arguments (this is confusing in the derived classes). If they really need them, then you have to make different functions with different argument lists (see for example Dmp\+Contextual\+::train(), there are 6 of them for this reason).\end{DoxyItemize}
\hypertarget{page_design_sec_naming}{}\subsection{Naming convention}\label{page_design_sec_naming}
I mainly follow the following naming style\+: \href{http://google-styleguide.googlecode.com/svn/trunk/cppguide.xml#Naming}{\tt http\+://google-\/styleguide.\+googlecode.\+com/svn/trunk/cppguide.\+xml\#\+Naming}

Notes\+: \begin{DoxyItemize}
\item Members end with a \+\_\+, i.\+e. {\ttfamily this\+\_\+is\+\_\+a\+\_\+member\+\_\+}. (Exception\+: members in a P\+O\+D (plain old data) class, which are public, and can be accessed directly) \item I also use this convention\+: \href{http://google-styleguide.googlecode.com/svn/trunk/cppguide.xml#Access_Control}{\tt http\+://google-\/styleguide.\+googlecode.\+com/svn/trunk/cppguide.\+xml\#\+Access\+\_\+\+Control} \item Abbreviation is the root of all evil! Long variable names are meaningful, and thus beautiful.\end{DoxyItemize}
Exceptions to the style guide above\+: \begin{DoxyItemize}
\item functions start with low caps (as in Java, to distinguish them from classes) \item filenames for classes follow the classname (i.\+e. Camel\+Cased)\end{DoxyItemize}
\hypertarget{page_design_Serialization}{}\subsection{Serialization}\label{page_design_Serialization}
See \hyperlink{page_serialization}{Serialization} 